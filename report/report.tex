\documentclass[a4paper,12pt]{article} 
\usepackage[ngerman]{babel}
\usepackage{graphicx}
%\usepackage{amsmath}
\usepackage{listings}
\usepackage{color}

\definecolor{gbred}{RGB}{251,73,52}
\definecolor{gbyellow}{RGB}{250,189,47}
\definecolor{gborange}{RGB}{254,128,25}
\definecolor{gbpurple}{RGB}{211,134,155}
\definecolor{gbgreen}{RGB}{184,187,38}
\definecolor{gbgray}{RGB}{146,131,116}

\lstset{
	frame=tb,
	language=Python,
	aboveskip=3mm,
	belowskip=3mm,
	showstringspaces=false,
	columns=flexible,
	basicstyle={\small\ttfamily},
	numbers=left,
	numberstyle=\tiny\color{gbgray},
	keywordstyle=\color{gbred},
	commentstyle=\color{gbgray},
	stringstyle=\color{gbyellow},
	emph={len, str, float, print},
	emphstyle=\color{gborange},
	breaklines=true,
	breakatwhitespace=true,
	tabsize=4
}

\begin{document}

{\Large\bf Autonomes Automobil}

\bigskip


{\bf  Abstract}

\bigskip


In diesem Bericht geht es um ... ("Ubersetzung erst, wenn der englische Text gut ist)

\medskip

{\it This report is about the creation of a self-driving model car using Python on a Raspberry Pi.}

Full-size self-driving cars (also known as autonomous cars) are currently in focus of research.
We are very interested in this topic and wanted to try this on our own.
For practical reasons, we used a model car and not a full-size one.
Our intention was to program the car, so it can go through an obstacle course without crashing.
We did not try to make the car safe for road traffic.

The development was both physically and computable.
We had to modify the available car and solder a few cables, and we had to program in Python.

In the end, we did not reach our initial goal.
But the car can at least drive straight ahead and stop immediately in front of a wall.
One could use the results of this project to make a car capable of driving an obstacle course.

\newpage


{\Large\bf Autonomes Automobil}

\medskip

Quentin Kniep, Timo Redweik, Nicolas Schmitt

\medskip

Friedrich-Wilhelm-Gymnasium K"oln, Severinstr. 241, 50676 K"oln

\medskip

5. April  2015

\medskip

{\bf  Abstract}
{\small We solve all problems. No questions remain.}

\medskip

{\bf  Zusammenfassung}
{\small Wir l"osen alle Probleme. Es bleiben keine Fragen.}

\medskip

{\bf  Keywords:}
{\small Gl"uck, Geld, Sorgenfreies Leben}

\bigskip


\section{Einf"uhrung}\label{sec1}

Ziel unseres Projekts war es, ein vorhandenes ferngesteuertes Modellautomobil (RC) so umzubauen, dass es selbst"andig Hindernissen ausweichen kann.
Es sollte einen zuf"alligen Hindernisparcours durchfahren k"onnen, ohne irgendwo anzusto"sen.

Wir sind auf dieses Projekt gekommen, weil wir uns sehr f"ur Technik interessieren.
Wir wollten eine Mischung aus Informatik und Anwendung; es sollte au"serdem leicht umsetzbar sein mit Aussicht auf Erfolg.
Wir sahen in dem Projekt zwar keine praktische Anwendungsm"oglichkeit, aber eine spannende Herausforderung.

Unser grundlegendes Konzept lautet wie folgt:
Wir brauchen auf jeder Seite des Autos einen Sensor, der den Abstand zum n"achstliegenden Objekt messen kann.
In das Auto m"ussen wir einen kleinen Computer einsetzen, der die Daten der Sensoren verarbeitet und den Motor steuert.
Diese "Uberlegung ist in Abb.~\ref{Fig1} dargestellt (S = Sensor, M = Motor).

\begin{figure}[h]
	\centering
	\includegraphics[width=12cm]{./media/overview.png}
	\caption{Konzept}
	\label{Fig1}
\end{figure}

Um sicherzugehen, welche Bauteile man braucht, und, wie sie alle zusammen funktionieren, haben wir im Internet unter [Quelle] nachgeschaut.
Wie wir uns dann entschieden haben, ist unter [Abschnitt Hardware] genauer dargestellt.

Wir haben die meiste Zeit zu dritt zusammen gearbeitet, f"ur die Dokumentation hat allerdings jeder einen Teilaspekt bearbeitet.
[Aufteilung]
Im folgenden Hauptteil haben wir zuerst die Hardware, dann die Software behandelt.

\section{Hauptteil}\label{sec2}

Zuerst haben wir das Problem analysiert und das Arbeitsprogramm erstellt.

\subsection{Hardware}\label{sec2.1}

Der erste Arbeitsunterpunkt wurde so behandelt und folgender L"osung zugef"uhrt.

\begin{equation}
	E = mc^2 ,
\end{equation}

die der Kollege Albert vom Patentamt Bern \cite{Alb05} auch schon beschrieben hat.

Diese "Uberlegung ist in Abb.~\ref{Fig2} dargestellt.

\begin{figure}[h]
	\centering
	\includegraphics[width=10cm]{./media/circuit_general.png}
	\caption{grundlegender Schaltplan}
	\label{Fig2}
\end{figure}

Die wichtigsten Teilaspekte haben wir in Tabelle~\ref{Tab1} zusammengefasst.

\begin{table}[h]
	\centering
	\begin{tabular}{|c|c|c|}
	\hline
		Gl"uck & Geld  & Sorgenfreies Leben  \\ \hline
		Ja  & Ja & Jaja \\ \hline
	\end{tabular}
	\caption{Sinnfindung}
	\label{Tab1}
\end{table}

\subsection{Software}\label{sec2.2}

F"ur die Steuerung des Autos haben wir ein Programm geschrieben, welches auf dem, von uns im Auto eingebauten, Raspberry Pi ausgef"uhrt wird.
Bei der Auswahl der Programmiersprache konnten wir uns, auf Grund unseres Vorwissens und der technischen Limitierung des Raspberry Pi, lediglich zwischen Python und C++ entscheiden. Wir haben uns dabei dann f"ur Python entschieden.
Vorallem auf Grund der einfachen Syntax, welche uns schnellere "Anderungen am Programm-Code erlaubt.

Die grundlegende Aufgabe des Programms ist es zuerst Messungen, mittels der Ultraschallsensoren durchzuf"uhren, diese Messungen auszuwerten, und dann entsprechende Signale an die Motoren des Autos zu senden.

\medskip

Zuerst definieren wir ein paar Konstanten, die in dem Programm h"aufiger verwendet werden, etwa den Umrechnungsfaktor mit dem wir aus der Zeit, die das Ultraschallsignal braucht, die Enfernung des Objektes berechnen k"onnen und umgekehrt.
Diesen Wert haben wir zuerst durch einen einfachen Versuch herausfinden m"ussen. Wir haben Objekte in bestimmten Entfernungen des Autos gestellt, und Messungen durchgef"uhrt.
Dies haben wir f"ur Objekte in verschiedenen Entfernungen durchgef"uhrt, sodass wir schlie"slich durch den Quotienten aus Strecke und Zeit den Umrechnungsfaktor errechnen konnten.
Oder auch die von uns festgelegte maximale Abweichung die ein Wert haben darf ohne als Fehlerwert zu gelten.
Und auch die Nummern der von uns verwendeten Pins am Raspberry Pi speichern wir in Konstanten, damit wir diese nachher einfacher abrufen k"onnen und die jeweilige Zahl nur ein einziges mal im Code vorkommt.
Somit muss bei einem Umstecken der Verbindungen nur eine Zahl ge"andert werden.

Au"serdem deklarieren wir einige Variablen, die wir verwenden k"onnen um w"ahrend der Ausf"uhrung des Programms Werte zwischenzuspeichern.
Die wichtigsten eieser Variablen sind unsere vier Listen.
In der Liste RESULT speichern wir die letzten Werte, die die Ultraschallsensoren gemessen haben, das hei"st die Zeiten zwischen Senden des Ultraschallsignals und erneuten Empfangen desselben, und den dazugeh"origen Zeitpunkt, zu dem die Messung stattgefunden hat.
In dieser Liste sind immer nur die letzten Messwerte gespeichert, bis sie auf Fehler "uberpr"uft wurden.
Wenn sie nach der "Uberpr"ufung als Fehlerwert gelten, d.h. eine zu gro"se Abweichung von dem vorigen Messwert haben, kommen sie in unsere Liste WDATA.
In dieser werden die Werte dann erneut f"ur eine weitere "Uberpr"ufung zwischengespeichert.
Bei dieser werden die letzten Messwerte je mit den Werten aus der vierten Liste verglichen.
Das hei"st es wird letztendlich "uberpr"uft ob sich die letzten Messwerte, nachdem ein Fehler erkannt wurde, stark genug "ahneln um doch als richtige Werte zu gelten.
Sobald die Werte nach einer "Uberpr"ufungen als richtige Werte angesehen werden, werden die Messwerte in die Liste DATA "uberf"uhrt und die dazugeh"origen Zeitpunkte in die Liste TIME.
Die Werte aus diesen beiden Listen k"onnenn letztendlich in Entfernungen umgerechnet werden und so weiterhin verwendet werden, etwa f"ur die Berechnung der Geschwindigkeit oder f"ur das Einleiten eines Bremsvorganges, wenn das Hindernis dem Auto zu nahe kommt.
Auf Grund von Fehlerwerten, die nicht als Messwerte aufgennommen werden, sind die Messungen nicht zwangsl"aufig in gleichm"a"sigen Zeitabst"anden. In diesen F"allen wird der Zeitpunkt der Messung ben"otigt um bestimmte Berechnungen, wie etwa die Errechnung der Geschwindigkeit, durchzuf"uhren.

\medskip

Beim Ausf"uhren des Programms wird zuerst unsere Hauptmethode aufgerufen, innerhalb dieser werden dann der Reihe nach einige spezialisierte Funktionen aufgerufen, welche sich jeweils mit einem Arbeitsschritt befassen.
Etwa die Methode measure(i), welche eine Messung mit dem Sensor i durchf"uhrt und das Ergebnis abspeichert.
Wobei i die Nummer des Sensors ist (wir haben diese von 0 bis 3 durchnummeriert).
Diese Funktionen werde ich im Folgenden einzeln erl"autern.

\medskip

\textbf{setup()}
\lstinputlisting[language=Python, firstline=30, lastline=46]{../main.py}
Diese Funktion wird als erstes ausgef"uhrt wenn das Hauptprogramm gestartet wird.
Sie ist daf"ur zust"andig alle Pins wieder auf ihren Ausgangszustand zur"uckzusetzen, d.h. die Pins auf OUT bzw. IN zu stellen und die angelegte Spannung der OUT-Pins auf 0V zu setzen.
Au"serdem werden die Einstellungen der GPIO Library gesetzt.
Die GPIO Library haben wir als Schnittstelle zwischen unserem Programm und den Pins des Raspberry Pi gew"ahlt.
Durch Funktionsaufrufe an diese Library k"onnen wir ablesen ob an den IN-Pins eine Spannung von 3,5 V anliegt und wir k"onnen an den OUT-Pins selbst eine Spannung von 3,5 V anlegen.

\medskip

\textbf{driveForward()}
\lstinputlisting[language=Python, firstline=143, lastline=149]{../main.py}
Beim Aufruf dieser Funktion f"angt das Auto anzufahren.
Dies wird durch eine "Anderung der Spannungen an den Pins, die "uber das Relay am Antriebsmotor angeschlossen sind, erreicht.
Diese "Anderung sorgt letztendlich daf"ur, dass ein geschlossener Stromkreis zwischen dem Akku des Autos und dem Antriebsmotor entsteht, und dass eine positive Spannung an diesem anliegt, was daf"ur sorgt, dass das Auto beginnt vorw"arts zu beschleunigen.

\medskip

\textbf{measure(i)}
\lstinputlisting[language=Python, firstline=54, lastline=81]{../main.py}
Der Parameter i in dieser Funktion beschreibt den Sensor, mit dem gemssen werden soll.
Dabei handelt es sich bei i um eine Zahl von 0 bis 3.
Zuerst f"uhrt diese Funktion eine Messung mit dem angegebenen Sensor durch.
Danach wird der erhaltene Messwert in der Liste RESULT abgespeichert.
Um eine Messung mit dem Ultraschallsensor zu starten muss man diesem zuerst "uber den daran angeschlossenen OUT-Pin einen elektrischen Impuls, der Dauer 10 µs und Spannung 3,5 V, geben.
Dann wartet unser Programm auf eine R"uckmeldung durch den Sensor, dies Erfolgt durch einen "ahnlichen elektrischen Impuls, den wir "uber den IN-Pin, der mit dem Sensor verbunden ist, empfangen k"onnen.
Den Zeitpunkt zu dem wir diesen Impuls empfangen haben speichern wir zwischen.
Da dieser genau dem Zeitpunkt entspricht zu dem das Ultraschallsignal gesendet wurde.
Entsprechend dem ersten kommt dann, bei Empfangen des Ultraschallsignals, ein weiterer Impuls, den wir erneut empfangen und dessen Zeitpunkt abspeichern.
Sodass wir schlie"slich durch die Differenz der beiden Zeitpunkte die Zeit herausfinden, die das Ultraschallsignal gebraucht hat, f"ur Hin- und R"uckweg.
Hierbei haben wir au"serdem einen Schutz eingebaut, der das Programm davor sch"utzt in einer Messung unendlich lang auf den zweiten Impuls zu warten.
Und zwar wird das Warten auf den zweiten Impuls nach 25 ms abgebrochen.
Falls dies passiert, wird als gemessener Zeitwert -1 abgespeichert, dies behandeln wir im Programm allgemein als fehlgeschlagene Messungen.

\medskip

\textbf{check\_results(i)}
\lstinputlisting[language=Python, firstline=108, lastline=136]{../main.py}
Bereits bei den ersten Tests ist uns aufgefallen, dass relativ h"aufig fehlerhafte Werte auftreten.
Deshalb war uns schon ziemlich fr"uh klar, dass wir eine Funktion brauchen, die die Messwerte auf ihre Richtigkeit "uberpr"uft.

\section{Fazit und Ausblick}\label{sec3}

Wir haben es geschafft, dass das Auto geradeaus fahren kann.
Das ist ein Teilerfolg, aber das urspr"ungliche Ziel (einen Parkours zu fahren) haben wir nicht erreicht.

Besondere Probleme haben uns die Kurven bereitet. ...

Au"serdem hatten wir Schwierigkeiten, bei dem immer komplexer werdenden Programm den "Uberblick zu behalten.
Wir haben den Aufwand und die Komplexit"at des Programms zuerst untersch"atzt.
Wir haben mehrmals versucht, einen hinreichenden Logfile zu implementieren; letztendlich hat es aber nicht gereicht, um die aufgetretenen Fehler zu identifizieren.

Wir haben nicht verstanden, wof"ur die uns empfohlenen [siehe ...] Widerst"ande gut sein sollen.
Vielleicht hat dieser Umstand auch einige Probleme verursacht, zum Beispiel bei den Messungen der Ultraschallsensoren?

Die hardwareseitige Entwicklung lief weitgehend zufriedenstellen.
Wir haben schnell den Aufbau und die Funktionsweise des Autos verstanden und konnten alle unsere "Uberlegungen umsetzen.

Wenn wir mehr Zeit h"atten, w"urden das Programm komplett neu schreiben und besonders auf ausreichende Logfile-Ausgaben achten.
Danach k"onnte man nocheinmal versuchen, das Auto Kurven fahren zu lassen.

\bigskip


{\large\bf Danksagung}

\medskip

blabla

\bigskip


{\large\bf Erkl"arung der Eigenst"andigkeit}

\medskip

blabla

\bigskip


\begin{thebibliography}{99}
	\itemsep-2pt \small
	\bibitem{Alb05} A. Einstein, Annalen der Physik {\bf 18}, 639 (1905).
	\bibitem{Alb06} A. Einstein, Annalen der Physik {\bf 18}, 639 (1906).
\end{thebibliography}


\end{document}
